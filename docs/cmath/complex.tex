\documentclass[11pt,a4paper,openany,oneside]{extreport}
\usepackage[T1]{fontenc}
\usepackage {amssymb}
\usepackage {amsmath}
\usepackage [cp1250]{inputenc}
\usepackage [polish]{babel}
\usepackage {graphicx} %mo�liwo�� zamieszczania obrazk�w
\usepackage {graphics}
\author{mateusz}
\begin{document}

\titlepage

This document is a short description of how complex arithmetic was implemented in fXaoS project. To provide 
as fastest complex math as possible, all expressions given in this paper have been implemented in assembly language using 
FPU operations. We use algebraical form of complex numbers given by equation
\begin{equation} 
z = a + \imath b \label{eq:cnalgebra}\text{.}
\end{equation}
However to obtain some of expressions we needed to use other representation :

\begin{description} 
\item[geometric representation of complex number]
 \begin{displaymath}
  \begin{array}{l}
  z = \|z\| \cdot \left( \cos\theta + \imath \sin\theta \right) \\
  \|z\| = \sqrt{z_\Re^2+z_\Im^2}=\sqrt{a^2+b^2} \\
  \theta = \arctan{\frac{z_\Im}{z_\Re}}=\arctan{\frac{b}{a}}\\
  \end{array}
 \end{displaymath}
\item[Euler's exponential form of complex number]
 \begin{displaymath}
  \begin{array}{l}
  e^{\imath\theta} = \cos\theta + \imath \sin\theta \\
  z = \|z\|\cdot e^{i\theta} \text{.}  
  \end{array}
 \end{displaymath}
\end{description}

We also have to remind some of complex number identities, like Euler's formulas
\begin{equation} 
 \begin{array}{rl}
   \sin{\theta} = \frac{1}{2\imath}\left( e^{\imath \theta}-e^{-\imath \theta} \right) &
   \cos{\theta} = \frac{1}{2}\left( e^{\imath \theta}+e^{-\imath \theta} \right)
 \end{array}
 \end{equation}
and their analogy for real
 \begin{equation}
  \begin{array}{rl}
  \sinh{x} = \frac{1}{2\imath}\left( e^{x}-e^{-x} \right) &
  \cosh{x} = \frac{1}{2}\left( e^{x}+e^{-x} \right) \text{.}
  \end{array}
 \end{equation}

Now you can do some math transformation to get what we want...

first, complex exponential of complex number
 \begin{displaymath}
  e^z = e^{a+\imath b} = e^a\cdot e^{\imath b} = e^a\cdot\left( \cos b +\imath \sin b \right)
 \end{displaymath}

complex natural logarithm of complex number
 \begin{displaymath}
  \ln{z} = \ln{ \|z\| \cdot e^{\imath \theta} } = \ln{\|z\|}+\imath \theta
 \end{displaymath}

complex logarithm (any base) of complex number
 \begin{displaymath}
  \log_n{z} = \frac{\ln z}{\ln n} = \frac{ \ln{\|z\|} }{ \ln n}+\imath\frac{ \theta }{ \ln n}= \ln_n{\|z\|} + \imath\frac{ \theta }{ \ln n}
 \end{displaymath}

using Euler formulas for complex argument we can obtain
 \begin{displaymath}
  \sin{\left(\imath \theta\right)} =  \frac{1}{2\imath}\left( e^{\imath (\imath\theta)}-e^{-\imath (\imath\theta)} \right) 
		=  \frac{1}{2\imath}\left( e^{-\theta}-e^{\theta} \right) = \imath \sinh{\theta} \\
 \end{displaymath}
 \begin{displaymath}
    \cos{\left(\imath \theta \right)} =  \frac{1}{2}\left( e^{\imath (\imath\theta)}+e^{-\imath (\imath\theta)} \right) 
	= \frac{1}{2}\left( e^{-\theta}+e^{\theta} \right) = \cosh{\theta}
 \end{displaymath}
 Next we can calculate other trigonometric funcitons
 \begin{displaymath}
  \tan{\left(\imath \theta \right)} =  \imath \tanh(\theta)
 \end{displaymath}
 
and by setting $\theta \rightarrow -\theta$ in above equations, we can get <immidately> hyperbolical functions
 \begin{displaymath}
  \begin{array}{l}
  \sinh(\imath \theta) = \imath \sin(\theta) \\
  \tanh(\imath \theta) = \imath \tan(\theta) \\
  \cosh(\imath \theta) = \cos(\theta) \\
  \end{array}
 \end{displaymath}
 
using trigonometrical realtions for sine and cosine we can get sine and cosine for complex arguments
 \begin{displaymath}
  \begin{array}{rl}
  \sin{z} = \sin{(a+\imath b)} = & \sin{(a)}\cos{(\imath b)} + \cos{(a)}\sin{(ib)} = \\
   \ & \ \\
   =& \sin{(a)}\cosh{(b)} + \imath \cos{(a)}\sinh{(b)} \\
  \end{array}
 \end{displaymath}
and
 \begin{displaymath}
  \begin{array}{rl}
  \cos{z} = \cos{(a+\imath b)} = & \cos{(a)}\cos{(\imath b)} - \sin{(a)}\sin{(ib)} = \\
   \ & \ \\
   =&  \cos{(a)}\cosh{(b)} - \imath \sin{(a)}\sinh{(b)} \\
  \end{array}
 \end{displaymath}
and also tangent function
 \begin{displaymath}
  \begin{array}{rl}
  \tan{z} = &\tan{(a+\imath b)} = \frac{ \tan(a)+\tan(\imath b) }{ 1 - \tan(a)\tan(\imath b) } = \frac{ \tan(a)+\imath \tanh(b) }{ 1 - \imath \tan(a)\tanh(b) } = \\
   \ & \ \\
   = & \frac{1}{1 + \tan^2(a)\tanh^2(b) }\left[(\tan(a)+\imath \tanh(b))(1+\imath \tan(a)\tanh(b))  \right]=\\
   \ & \ \\
   = & \frac{\tan(a)- \tan(a)\tanh^2(b)}{1 + \tan^2(a)\tanh^2(b) } + \imath \frac{\tan^2(a)\tanh(b) + \tanh(b)}{1 + \tan^2(a)\tanh^2(b) } \\
  \end{array}
 \end{displaymath}
 
there is a faster way to calculate tangent function, using its definition. 
after implementing tangent function, i came up with very simple idea to use the its definition. instead
of trygonometric identity used in previous equation. 
 \begin{displaymath}
  \begin{array}{rl}
  \tan{z} = &\frac{\sin{z}}{\cos{z}} = \frac{\sin{(a)}\cos{(\imath b)} + \cos{(a)}\sin{(ib)}}{\cos{(a)}\cos{(\imath b)} - \sin{(a)}\sin{(ib)}} = \\
   \ & \ \\
   = & \frac{\left[\sin{(a)}\cos{(\imath b)} + \cos{(a)}\sin{(ib)}\right]\left[ \cos{(a)}\cos{(\imath b)} + \sin{(a)}\sin{(ib)} \right]}{ \left[ \cos{(a)}\cos{(\imath b)} - \sin{(a)}\sin{(ib)} \right]\left[ \cos{(a)}\cos{(\imath b)} + \sin{(a)}\sin{(ib)} \right] } =\\
   \ & \ \\
   = & \frac{ \sin{a}\cos{a}\cosh^2{b} - \cos{a}\sin{a}\sinh^{b}+\imath\left[ \sin^2{a}\cosh{b}\sinh{b}+\cos^2{a}\cosh{b}\sinh{b} \right]}{ \cos^2{a}\cosh^2{b}+\sin^2{a}\sinh^2{b} }=\\
   \ & \ \\
   = & \frac{ \frac{1}{2} \sin{2a} + \imath \sinh{2b} }{ \cos^2{a}+\sinh^2{b} } = \frac{ \sin{2a} + \imath \sinh{2b} }{ 2\cos^2{a}+2\sinh^2{b} } = \frac{ \sin{2a} + \imath \sinh{2b} }{ 2\cos^2{a}-1 + 2\sinh^2{b}+1 } = \\
   \ & \ \\
   = & \frac{ \sin{2a} + \imath \sinh{2b} }{ \cos{2a} + \cosh{2b} } \\
  \end{array}
 \end{displaymath}
 i've seen that eqation being called Abramowitz formula. it is about 15\% faster than the previous one. analogically we can get cotangent function of form :
 \begin{displaymath}
  \cot{z} = \frac{ \sin{2a} - \imath \sinh{2b} }{ \cosh{2b} - \cos{2a} }
 \end{displaymath}
  
in next step we will <wyprowadzic> hyperbolic trigonometry functions, using some well known realations (similar to those used previously).
hyperbolic sine
 \begin{displaymath}
  \begin{array}{rl}
  \sinh{z} = & \sinh{(a+\imath b)} = \sinh{(a)}\cosh{(\imath b)} + \cosh{(a)}\sinh{(ib)} = \\
  \ & \ \\
  = & \sinh{(a)}\cos{(b)} + \imath \cosh{(a)}\sin{(b)}
  \end{array}
 \end{displaymath}
further we get hyperbolic cosine
 \begin{displaymath} 
  \begin{array}{rl}  
  \cosh{z} = & \cosh{(a+\imath b)} = \cosh{(a)}\cosh{(\imath b)} + \sinh{(a)}\sinh{(ib)} = \\
  \ & \ \\
  = & \cosh{(a)}\cos{(b)} + \imath \sinh{(a)}\sin{(b)}
  \end{array}
 \end{displaymath}
and also hyperbolic tangent
 \begin{displaymath}
  \begin{array}{rl}
	\tanh{z} = & \tanh{(a+\imath b)} = \frac{ \tanh(a) + \tanh(\imath b) }{1+ \tanh(a)\tanh(\imath b) } = \frac{ \tanh(a) + \imath \tan(b) }{1+ \imath \tanh(a)\tan(b) } = \\ 
   \ & \ \\
   = & \frac{1}{1 - \tanh^2(a)\tan^2(b) }\left[ (\tanh(a)+\imath \tan(b))(1-\imath \tanh(a)\tan(b)) \right]=\\
   \ & \ \\
   = & \frac{\tanh(a) + \tanh(a)\tan^2(b)}{1 - \tanh^2(a)\tan^2(b) } + \imath \frac{\tan(b)-\tan(b)\tanh^2(a)}{1 - \tanh^2(a)\tan^2(b) } \\
  \end{array}
 \end{displaymath}

like in case of tangent function, also now we can get better expression for hyperbolic tangent. the derivation of that formula is almost identical to tangent formula derivation. using it to hyperbolic cotangent funciton we can get :
\begin{displaymath}
  \begin{array}{rl}  
   \tanh(z) = & \frac{ \sinh{2a} + \imath \sin{2b} }{ \cosh{2a} + \cos{2b} } \\
   \ &\ \\
   \coth(z) = & \frac{ \sinh{2a} - \imath \sin{2b} }{ \cosh{2a} - \cos{2b} } \\   
  \end{array}
\end{displaymath}

in next section we will show complex power functions. where are three different ways to evaluate power functions in complex plane, we get one way for complex
number to the integer power (de Moivre relation), other for complex number to the real power and quite similar complex number to the complex number.
complex number to integer power (de Moivre)
 \begin{displaymath}
 z^n = \|z\|^n\left[ \cos\left(n\theta\right) + \imath \sin\left(n\theta\right) \right]
 \end{displaymath}

complex number to real power
 \begin{displaymath}
 z^r = \exp{(r\cdot \ln z)}=\exp{(r\ln\|z\| + \imath r\theta)}=e^{r\ln\|z\|}[\cos{(r\theta)}+\imath \sin{(r\theta)}]
 \end{displaymath}

complex number to complex power
 \begin{displaymath}
  \begin{array}{rl}
 z^{z_1} = &z^{c+\imath d}=\exp{(z_1\cdot \ln z)}=\exp{(z_1\cdot[\ln\|z\|+\imath \theta])}=\\
  = & \exp{(c\ln\|z\|-d\theta+\imath[c\theta+d\ln\|z\|])}= \\
  = & e^{c\ln\|z\|-d\theta}[\cos(c\theta+d\ln\|z\|)+\imath \sin(c\theta+d\ln\|z\|)]
  \end{array}
 \end{displaymath}

The last case is when we want to calculate integer (or real) number to the complex power
\begin{displaymath}
 \begin{array}{rl}
 n^z = & n^{a+\imath b} = \exp( z \ln n ) = \exp(a\ln n + \imath b\ln n) = \\
 = & n^a \left( \cos(b \ln n) + \imath \sin(b \ln n) \right )
 \end{array}
\end{displaymath}

We should also be able to computate complex numbers roots of any given order. using de Moivre relation we can 
easily derive formula for calculating Nth order root. 

\begin{displaymath}
	\sqrt[n]{z} = \sqrt[n]{\|z\|}\left[ \cos\left( \frac{\theta+2\pi i}{n} \right) + \imath \sin\left( \frac{\theta+2\pi i}{n} \right) \right]
\end{displaymath}

where $i=0..n-1$, writing computer implementation of this formula we have to remember that complex numbers have exactly N 
solutions of Nth order root problem. we should also tread as a special case square root function, 
using previous formula will can derive squre root formula.

\begin{displaymath}
	\sqrt{z} = \sqrt{\|z\|}\left( \cos\frac{\theta}{2}+\imath \sin\frac{\theta}{2}\right ) =\sqrt{ \frac{\|z\| + a}{2} } \pm \imath \sqrt{ \frac{ \|z\| - a}{2} }
\end{displaymath}

now if we put to this eqaution this two relations
\begin{displaymath}
 \begin{array}{l}
 \sin \frac{\theta}{2} = \sqrt{ \frac{1-\cos \theta }{2} } \\
 \cos \frac{\theta}{2} = \sqrt{ \frac{1+\cos \theta }{2} } \\
 \end{array}
\end{displaymath}

we will get the complex square root function
\begin{displaymath}
	\sqrt{z} =\sqrt{ \frac{\|z\| + a}{2} } \pm \imath \sqrt{ \frac{ \|z\| - a}{2} }
\end{displaymath}

where sign of imaginary part should be the same as sign of $b$.

 \end{document}
